\title{A Sample Document for the usages of \textsf{abbrev} package}
\author{Thai Son Hoang\\ETH Zurich\\\texttt{<htson at inf dot ethz dot
    ch>}}

\begin{document}
\maketitle

% Create some abbreviation macros
\newabbrev{SME}{Small and Medium-sized Enterprise}
\newabbrev[randd]{R\&D}{Research \& Development}
% This macro will not be used and hence will not appear in the list of abbreviations.
\newabbrev{OECD}{Organisational for Economic and Cooperation Development}

First occurrences of the macros will be expanded, such \SMEs and \randd{}.

\SME: second appearance of SME(s) and will be abbreviated. 

\randds: second appearance of R\&D(s) and will be abbreviated.

\SMEs: third appearance of SME(s) and will be abbreviated.

\randds: third appearance of R\&D(s) and will be abbreviated.

\SMEs: fourth appearance of SME(s) and will be abbreviated.

\randd: fourth appearance of R\&D(s) and will be abbreviated.

\section{Section 2}
\label{sec:section-2}

\resetabbrev

\SMEs: first appearance of SME(s) and will be expanded.

\SME: second appearance of SME(s) and will be abbreviated. 

\randd: first appearance of R\&D(s) and will be expanded

\randds: second appearance of R\&D(s) and will be abbreviated.

\SMEs: third appearance of SME(s) and will be abbreviated.

\randds: third appearance of R\&D(s) and will be abbreviated.

\SMEs: fourth appearance of SME(s) and will be abbreviated.

\randd: fourth appearance of R\&D(s) and will be abbreviated.

% Print the list of used abbreviations
\printnomenclature

\end{document}
%%% Local Variables: 
%%% mode: latex
%%% TeX-master: "sample-abbrev"
%%% End: 

\title{A Sample Document for the usages of \textsf{abbrev} package}
\author{Thai Son Hoang\\ETH Zurich\\\texttt{<htson at inf dot ethz dot
    ch>}}

\begin{document}
\maketitle

% Create some abbreviation macros
\newabbrev{SME}{Small and Medium-sized Enterprise}
\newabbrev[randd]{R\&D}{Research \& Development}
% This macro will not be used and hence will not appear in the list of abbreviations.
\newabbrev{OECD}{Organisational for Economic and Cooperation Development}

First occurrences of the macros will be expanded, such as \SMEs and \randd.

Subsequent occurrences of the macros will be abbreviated, such as \SME
and \randds.

The macros can be \emph{reset}.
\resetabbrev

Afterwards, the behaviours of the abbreviations macros will be such
that they have never been used.

First occurrences of the macros will be expanded, such as \SMEs and \randd.

Subsequent occurrences of the macros will be abbreviated, such as \SME
and \randds.

The list of abbreviations can be printed.  Only the abbreviations used
will be added to the list.
% Print the list of used abbreviations
\printnomenclature

\end{document}
%%% Local Variables: 
%%% mode: latex
%%% TeX-master: "sample-abbrev"
%%% End: 
